The results are best shown in an ``injection table'' giving the amount of insulin to inject (in units) for a given glucose intake (in grams) and for a given desired change in glycemia (in mg/dL).
Table \ref{tab:injections} shows one such table for the ``normal'' state (average across activity, alcohol and sickness status) for a time threshold of 3 hours.
\textcolor{red}{Should I not distinguish ``normal'' and ``average'' then?}\\

It seemed that a time threshold of three hours gave the most reliable results, and so I only show those in Fig. \ref{fig:regressions} and Table \ref{tab:injections}.
Still, the predicted insulin injections were very odd, with (based on my own experience) very low doses for glucose intakes that I know should require more (e.g. $1.5$ units to increase glycemia by $50$ mg/dL when eating $65$ grams of glucose), and very high doses for other changes in glycemia which I know require much less insulin (e.g. $17.6$ units to decrease glycemia by $300$ mg/dL when eating nothing, Table \ref{tab:injections}). 
And so, although the exercise was valuable, I decide not to trust those results.