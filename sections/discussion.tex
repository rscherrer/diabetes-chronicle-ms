There are several reasons why the results are not trustworthy.
First, the data were collected in everyday life, so non-control conditions, with lots of confounding factors known to have an effect on blood sugar (exercise, stress, etc.).
Second, because of the non-control conditions, the chunks of time taken as data points were of varying durations, with some chunks spanning multiple meals and multiple activities until blood sugar stabilizes.
The increase in cumulative insulin and cumulative glucose (and exercise when applicable) during those periods may have introduced imprecisions in the data.
Third, there probably were mistakes in the estimation of the actual carbohydrates taken in, whether due to imprecise weighing or incorrect carbohydrate contents of the food items.\\

The solution seems to be to repeat this kind of analysis in controlled conditioned, that is, with much less variation in time span, and by removing as many confounding factors as possible.
Of course, this may affect the generality of the results regarding everyday life, but at least would provide a solid and reliable baseline for how much insulin is needed to absorb a given amount of sugar.
Those doses could then be slightly modified when under stress, alcohol, sick or when doing exercise.
That said, I noticed that the breakfast recorded in the data were especially consistent and precise, resembling controlled conditions.
\textcolor{red}{Maybe I could repeat my analysis on the same data but only looking at those and see if the predictions are more reliable.}
