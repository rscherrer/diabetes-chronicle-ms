Self-management of type 1 diabetes can be constraining.
One approach to alleviate those constraints is flexible insulin therapy, where a patient picks a dose of insulin based on how much carbohydrates they eat, rather than the more traditional approach of adapting meals to fixed, pre-decided amounts of insulin.
In order to implement flexible insulin therapy, however, the right ratios must be known, for that patient, of insulin injected per gram of glucose consumed.
This requires extensive data on changes in blood sugar, or glycemia, in response to various amounts of carbohydrates eaten and insulin injected.
Here, I aimed at performing this analysis on myself, as a patient with type 1 diabetes.
Using recorded data over a span of two months and equipped with a continuous glucose monitoring system (CGM), I compiled a dataset comprising instances of food eaten, glycemia checked and insulin injected.
I also considered additional factors known to affect blood sugar such as physical activity, alcohol use and infections.
Unfortunately, the regression analyses performed did not result in parameter estimates that I could trust, probably due to mistakes in estimations of carbohydrate contents but also messy everyday life as opposed to if I had collected data in more controlled conditions.
Anyway, that was an insightful exercise which I thought to write up as to not forget how I did it.